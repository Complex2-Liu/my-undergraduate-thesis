\chapter{引言}
\label{chap:introduction}
\fontsize{12bp}{14.4pt}

我在本学期选修了一门叫“基础代数几何”的课,
该门课的主要理论来自 \cite{textbook},
以复变函数为主要工具来研究代数曲线的各种性质.
首先, 代数曲线和 Riemann 面有极其紧密的关系:
正则化定理(c.f. \cref{thm:normalization}) 告诉我们在全纯同构的意义下,
每条不可约代数曲线唯一决定一个紧 Riemann 面,
于是我们可以讨论一条代数曲线的亏格.
在课上我们证明了特殊情形的亏格公式(\cite{textbook}, 第二章的定理 9.1):
设 $C$ 是一条次数为 $d$ 的不可约代数曲线,
有 $\delta$ 个奇点, 并且这 $\delta$ 个奇点都是普通二重点, 则有
\begin{equation}
\label{eq:genus-formula-multiple-two}
g = \frac{1}{2}(d-1)(d-2) - \delta.
\end{equation}
本文的核心目标为用类似的方法来证明一般情形的亏格公式.

在第二章, 我们会先介绍一些基本概念,
主要包括代数曲线和 Riemann 面的定义及基本性质.
代数曲线 $C$ 上的点可以分为两类: 光滑点和奇点.
其中光滑点是性质比较好的点, 因为这时我们总是能运用隐函数定理,
由此刻画 $C$ 在光滑点附近的性质.
但同样的方法却不能如法炮制的运用在奇点处,
所以我们必须另寻出路.
因为奇点的个数是有限的(c.f. \cref{prop:finiteness-of-singularity}),
所以要研究一个奇点, 我们需要把更多的注意力放在它的局部.
在第三章的第一节, 我们会发现(c.f. \cref{thm:Weierstrass-preparation}):
一个多项式的零点 $f(x,y)$ 局部来看是由一些 Weierstrass 多项式 $w(x,y)$ 的零点贡献.
于是, 问题转化为研究 Weierstrass 多项式的零点.
有了以上这么多的铺垫, 我们才能精确理解正则化定理的每一个细节.
在第三章的最后, 我们引入相交数的概念,
它能帮助我们更好地刻画一条直线是否与曲线 $C$ 相切于某点(c.f. \cref{prop:tangent-chara}).
同时, 相交数也是我们证明亏格公式的主要想法.

我们首先会把亏格公式推广到不局限于普通二重点,
而是任意重数的普通奇点, 在这种情形,
我们会得到一个同 \cref{eq:genus-formula-multiple-two} 一样,
只依赖于曲线次数 $d$ 和各奇点重数的亏格公式.
对于更一般的情形, 即奇点不一定全为普通奇点,
亏格公式会相对复杂, 详细的讨论见第四章的第二和第三节.
